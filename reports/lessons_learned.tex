\documentclass[12pt,a4paper,leqno]{report}

\usepackage[T1]{fontenc}
\usepackage[english]{babel}
\usepackage{amsthm}
\usepackage{amsfonts}
\usepackage{amsmath}
\usepackage{amssymb}
\usepackage{tikz}
\usepackage{listings}

\newcommand{\R}{\mathbb{R}}
\newcommand{\C}{\mathbb{C}}
\newcommand{\Q}{\mathbb{Q}}
\newcommand{\N}{\mathbb{N}}
\newcommand{\No}{\mathbb{N}_0}
\newcommand{\Z}{\mathbb{Z}}
\newcommand{\diam}{\operatorname{diam}}

\theoremstyle{plain}
\newtheorem{equa}[equation]{Equation}
\newtheorem{lem}[equation]{Lemma}
\newtheorem{prop}[equation]{Proposition}
\newtheorem{cor}[equation]{Corollary}

\theoremstyle{definition}
\newtheorem{defi}[equation]{definition}
\newtheorem{conj}[equation]{Conjecture}
\newtheorem{example}[equation]{Example}
\theoremstyle{remark}
\newtheorem{note}[equation]{Note}

\pagestyle{plain}
\makeatletter
\renewcommand{\@seccntformat}[1]{}
\makeatother
\setcounter{page}{1}
\addtolength{\hoffset}{-1.15cm}
\addtolength{\textwidth}{2.3cm}
\addtolength{\voffset}{0.45cm}
\addtolength{\textheight}{-0.9cm}

\graphicspath{ {./figures/} }

\title{Big Data - Lessons Learned}
\author{Tuomo Kareoja}
\date{\today}

\begin{document}

\maketitle

We cannot approach this task as a normal predictive modelling exercise
as it will not be useful or even possible to have the training data
in use in production and stop people from getting more credit when the model
tells us so. This is because we are a credit rating agency and not in command of
of actual crediting and data: that is a job of the credit card company.

Creating a predictive model might still be very useful though. After the
initial exploration of the data it is clear that the finding the defaulters
is complex job and if we just look at graphs we might not get far with understanding
the complex relationships at play. A good example of this is the fact that the bigger
the debt a customer has the less likely she is to default. This makes no sense of course
and it means that the correlation cannot be causation. It also means that to understand
the relation of debt to defaulting we must look at trough the interactions of multiple factors.
A model can automatically find these kinds of complex relations,
but we must make sure that we can get this info out of the model. Unfortunately this greatly
limits the kinds of models we can apply in this task.

The ultimate goal of our analysis should be to understand (and not just predict) how
different factors affect defaulting. With this information we could see how modifying the
credit limit would affect the predictions. We are in the end only interested in seeing how this
variable affects defaulting, because it is the one that we have control over. In the end we
want to find out for who and how much we should lower our credit rating and if there are
customers we should not have given credit at all.

\end{document}
